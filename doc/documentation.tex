\documentclass{article}
\nonstopmode

\title{Trajectory Optimization Framework}
\author{Johnathan Van Why}

\begin{document}
	\maketitle
	\section{Introduction}
		This framework is designed for the optimization of trajectories for robotic systems.
		It can handle underactuated and hybrid systems. In the future, it will likely have support for multiple
		"phases", or time periods in which specific dynamics are enforced.

		The framework makes use of the MATLAB Symbolic Toolbox to aid in problem formulation and manipulation.
		The actual optimization, however, is almost always numerical. By keeping the problem in a symbolic form, we
		allow for the implementation of a mode where the optimization algorithm and problem are formulated together, leading
		to a final solver optimized for the specific problem's structure.

		This document gives a basic description of how to use the framework, which
		will eventually be supplemented by code samples (there will soon be a sample program
		for the reworked interface in the repository). It also covers the architectural design of the framework
		and details on some of the most important algorithms contained within the framework.

	\section{Usage}
		At the moment, the interface to the optimizer is undergoing a re-design, so it is likely to change in the near future.
		This section will hopefully be kept up to date with changes.

		One thing that currently appears to be invariant is the usage of a central structure, called the "scenario" structure.
		The scenario structure contains all information necessary to run an optimization -- it is incrementally filled out by the optimizer
		as it completes the various stages of problem setup and optimization. The final scenario structure contains the results of the optimization,
		and should contain all information that may need to be saved for future use.
\end{document}
